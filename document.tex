\documentclass[conference]{IEEEtran}
\usepackage{amsmath}
\usepackage{graphicx}
\usepackage{hyperref}

\title{Exploring Anonymization Methods}
\author{\IEEEauthorblockN{Tianci Xie and Haochun Qi}}

\begin{document}
	
	\maketitle
	
	\begin{abstract}
		Anonymization is a critical technique in preserving privacy in datasets. This report explores the implementation and evaluation of two privacy-preserving methods: \textbf{k-anonymity} and \textbf{l-diversity}. Two different cases were examined: the first focused on iterative generalization and suppression techniques to achieve both k-anonymity and l-diversity, while the second case demonstrated an alternative anonymization approach with seemingly error-free implementation. 
	\end{abstract}
	
	\section{Introduction}
	In the era of big data, maintaining the privacy of individuals is of paramount importance. Data anonymization techniques aim to protect sensitive information while preserving the utility of datasets for analysis. This project focuses on two widely used anonymization techniques:
	\begin{itemize}
		\item \textbf{k-Anonymity}: Ensures that each individual is indistinguishable from at least $k-1$ others based on quasi-identifiers.
		\item \textbf{l-Diversity}: Extends k-anonymity by ensuring that sensitive attributes have at least $l$ distinct values within each group defined by quasi-identifiers.
	\end{itemize}
	
	\section{Methodology}
	Two cases were implemented to evaluate anonymization techniques.
	
	\subsection{Case 1: Iterative Generalization and Suppression}
	In this case, the dataset was generated with the following attributes:
	\begin{itemize}
		\item \textbf{Age}: Random integer between 20 and 80.
		\item \textbf{ZIP Code}: Random integer between 12340 and 12349.
		\item \textbf{Gender}: Random choice of ``Male'' or ``Female''.
		\item \textbf{Disease}: Random choice from \{``Flu'', ``Diabetes'', ``Cancer'', ``Hypertension''\}.
	\end{itemize}
	
	The following steps were applied:
	\begin{enumerate}
		\item \textbf{Generate Dataset}: Created a synthetic dataset with 100 records using a Python function that randomly assigns values to the attributes. The dataset ensures diversity in both quasi-identifiers and sensitive attributes.
		\item \textbf{k-Anonymity Check}: Grouped records based on quasi-identifiers (``Age'', ``ZIP Code'', ``Gender'') and computed the size of each group. The minimum group size determines the current k-anonymity level of the dataset.
		\item \textbf{l-Diversity Check}: Evaluated diversity within groups by calculating the number of unique values of the sensitive attribute (``Disease'') in each group. The minimum diversity across groups determines the l-diversity of the dataset.
		\item \textbf{Anonymization Techniques}:
		\begin{itemize}
			\item \textbf{Generalizing ZIP Code}: Applied a masking function that replaces the last two digits of the ZIP Code with asterisks (e.g., 12345 becomes 123**). This step reduces the granularity of location data.
			\item \textbf{Generalizing Age}: Reduced precision by grouping ages into 10-year intervals (e.g., 25 becomes 20, 67 becomes 60). This iterative process continues until k-anonymity is achieved.
			\item \textbf{Suppressing Rows}: For groups that fail the l-diversity requirement, rows were removed entirely to ensure that all remaining groups meet the diversity threshold.
		\end{itemize}
	\end{enumerate}
	
\subsection{Case 2: Alternative Anonymization Implementation}

This case used the UCI Adult Income dataset, a publicly available dataset with realistic data distributions. The dataset contains demographic and employment-related attributes. For this study, the following attributes were used: \begin{itemize} \item \textbf{Age}: Numeric values, generalized into decades for anonymization (e.g., 20-29, 30-39). \item \textbf{Workclass}: Categorical values, simplified to Public/Private'' for anonymization. \item \textbf{Gender}: Binary values (Male/Female). \item \textbf{Income}: Sensitive attribute, categorized as <=50K'' or ``>50K''. \end{itemize}

The following steps were applied for anonymization: \begin{enumerate} \item \textbf{Generalization of Quasi-Identifiers:} \begin{itemize} \item \textbf{Age}: Generalized into decades by truncating the last digit (e.g., 25 becomes 20). \item \textbf{Workclass}: Simplified into a single group, ``Public/Private,'' to reduce granularity. \end{itemize} \item \textbf{k-Anonymity Check:} Groups were formed based on quasi-identifiers (Age, Workclass, Gender), and the minimum group size was checked to ensure compliance with a specified $k$ value. \item \textbf{l-Diversity Check:} For each quasi-identifier group, the number of unique values in the sensitive attribute (Income) was counted. Groups failing to meet the $l$-diversity requirement were flagged for suppression. \item \textbf{Suppression:} Groups that failed the $k$-anonymity or $l$-diversity checks were suppressed by removing their records. \end{enumerate}

This approach ensured compliance with both $k$-anonymity and $l$-diversity while minimizing the loss of data utility. The transformations were straightforward and did not require iterative adjustments, as the dataset's structure was already conducive to anonymization.


\section{Results}

\subsection{Case 1 Results}
\begin{itemize}
	\item When $k=1, 2,$ or $3$, k-anonymity was seldomly unsuccessful. However, as $k$ increased to $4$ or higher, the failure rate rose significantly, with larger values of $k$ leading to greater failure rates.
	\item Once k-anonymity was achieved, l-diversity could always be satisfied as long as $l \leq 4$, which matches the number of unique types of sensitive attribute values.
\end{itemize}

\subsection{Case 2 Results}
\begin{itemize}
	\item After applying the specified anonymization methods, the dataset achieved $k=10$ and $l=2$.
	\item Since a pre-existing dataset and consistent anonymization techniques were used, the results remained deterministic and reproducible across trials.
\end{itemize}

\section{Discussion}
The results highlight the differing dynamics between the two cases. In Case 1, where synthetic data and iterative transformations were used, achieving higher values of $k$ became increasingly challenging due to the need for greater generalization and suppression. The variability in success rates reflects the inherent randomness in the synthetic dataset generation and the limited diversity of quasi-identifiers.

In contrast, Case 2 demonstrated a deterministic outcome due to the structured nature of the existing dataset and predefined anonymization methods. The achieved values of $k=10$ and $l=2$ indicate that the dataset was well-suited for anonymization without significant utility loss. This consistency underscores the importance of dataset structure and inherent diversity in facilitating anonymization.

Furthermore, the results from Case 1 showed that once k-anonymity was achieved, l-diversity could also be satisfied within reasonable limits. This reinforces the idea that achieving k-anonymity can lay a foundation for l-diversity, provided the sensitive attribute is sufficiently diverse.

\section{Conclusion}
This study explored the application of k-anonymity and l-diversity in two distinct scenarios. Case 1 involved synthetic data and iterative methods, revealing challenges in achieving higher k-anonymity values and highlighting variability due to dataset randomness. Case 2 utilized an existing dataset and a straightforward anonymization approach, yielding consistent and deterministic results.

The findings emphasize the trade-offs between generalization, suppression, and dataset utility. They also underscore the impact of dataset structure and inherent diversity on the effectiveness of anonymization methods. Future work could explore additional techniques, such as differential privacy, to address limitations and further enhance privacy protections.

	
\section*{References}
\begin{itemize}
	\item UCI Machine Learning Repository. Adult Data Set. \url{https://archive.ics.uci.edu/ml/datasets/adult}.
\end{itemize}

	
\end{document}

